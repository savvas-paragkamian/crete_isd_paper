%%
%% Copyright 2022 OXFORD UNIVERSITY PRESS
%%
%% This file is part of the 'oup-authoring-template Bundle'.
%% ---------------------------------------------
%%
%% It may be distributed under the conditions of the LaTeX Project Public
%% License, either version 1.2 of this license or (at your option) any
%% later version.  The latest version of this license is in
%%    http://www.latex-project.org/lppl.txt
%% and version 1.2 or later is part of all distributions of LaTeX
%% version 1999/12/01 or later.
%%
%% The list of all files belonging to the 'oup-authoring-template Bundle' is
%% given in the file `manifest.txt'.
%%
%% Template article for OXFORD UNIVERSITY PRESS's document class `oup-authoring-template'
%% with bibliographic references
%%

%%%CONTEMPORARY%%%
\documentclass[unnumsec,webpdf,contemporary,large]{oup-authoring-template}%
%\documentclass[unnumsec,webpdf,contemporary,large,namedate]{oup-authoring-template}% uncomment this line for author year citations and comment the above
%\documentclass[unnumsec,webpdf,contemporary,medium]{oup-authoring-template}
%\documentclass[unnumsec,webpdf,contemporary,small]{oup-authoring-template}

%%%MODERN%%%
%\documentclass[unnumsec,webpdf,modern,large]{oup-authoring-template}
%\documentclass[unnumsec,webpdf,modern,large,namedate]{oup-authoring-template}% uncomment this line for author year citations and comment the above
%\documentclass[unnumsec,webpdf,modern,medium]{oup-authoring-template}
%\documentclass[unnumsec,webpdf,modern,small]{oup-authoring-template}

%%%TRADITIONAL%%%
%\documentclass[unnumsec,webpdf,traditional,large]{oup-authoring-template}
%\documentclass[unnumsec,webpdf,traditional,large,namedate]{oup-authoring-template}% uncomment this line for author year citations and comment the above
%\documentclass[unnumsec,namedate,webpdf,traditional,medium]{oup-authoring-template}
%\documentclass[namedate,webpdf,traditional,small]{oup-authoring-template}

%\onecolumn % for one column layouts

%\usepackage{showframe}

\graphicspath{{Fig/}}

% line numbers
%\usepackage[mathlines, switch]{lineno}
%\usepackage[right]{lineno}

\theoremstyle{thmstyleone}%
\newtheorem{theorem}{Theorem}%  meant for continuous numbers
%%\newtheorem{theorem}{Theorem}[section]% meant for sectionwise numbers
%% optional argument [theorem] produces theorem numbering sequence instead of independent numbers for Proposition
\newtheorem{proposition}[theorem]{Proposition}%
%%\newtheorem{proposition}{Proposition}% to get separate numbers for theorem and proposition etc.
\theoremstyle{thmstyletwo}%
\newtheorem{example}{Example}%
\newtheorem{remark}{Remark}%
\theoremstyle{thmstylethree}%
\newtheorem{definition}{Definition}

\begin{document}

\journaltitle{The ISME Journal}
\DOI{DOI HERE}
\copyrightyear{2024}
\pubyear{2024}
\access{Advance Access Publication Date: Day Month Year}
\appnotes{Paper}

\firstpage{1}

%\subtitle{Subject Section}

\title[Crete's soil microbiome]{Exploring Crete's Macroecology based on its soil Microbial Interactome}

\author[1,2]{Savvas Paragkamian\ORCID{0000-0002-8508-2521}}
\author[3,4]{Johanna B. Holm\ORCID{0000-0002-7646-4085}}
\author[5]{Christina Pavloudi\ORCID{0000-0001-5106-6067}}
\author[6]{Haris Zafeiropoulos\ORCID{0000-0002-4405-6802}}
\author[1]{Christos A. Christakis\ORCID{0000-0002-7075-0996}}
\author[1]{Melanthia Stavroulaki\ORCID{0000-0003-4392-7159}}
\author[7]{Stephanie A. Yarwood\ORCID{0000-0003-4593-7908}}
\author[2]{Antonios Magoulas}
\author[1]{Panagiotis F. Sarris\ORCID{0000-0001-7000-8997}}
\author[2]{Giorgos Kotoulas\ORCID{0000-0002-4667-8533}}
\author[3]{Lynn Schriml\ORCID{0000-0001-8910-9851}}
\author[2,$\ast$]{Evangelos Pafilis\ORCID{0000-0001-5079-0125}}

\authormark{Paragkamian et al.}

\address[1]{\orgdiv{Department of Biology}, \orgname{University of Crete}, \orgaddress{\street{Voutes}, \postcode{70013}, \state{Crete}, \country{Greece}}}
\address[2]{\orgdiv{Institute of Marine Biology, Biotechnology and Aquaculture (IMBBC)}, \orgname{Hellenic Centre for Marine Research (HCMR)}, \orgaddress{\street{Gournes}, \postcode{71500}, \state{Crete}, \country{Greece}}}
\address[3]{\orgdiv{Institute for Genome Sciences}, \orgname{University of Maryland School of Medicine}, \orgaddress{\state{Baltimore}, \country{USA}}}
\address[4]{\orgdiv{Department of Microbiology and Immunology}, \orgname{University of Maryland School of Medicine}, \orgaddress{\state{Baltimore}, \country{USA}}}
\address[5]{\orgname{European Marine Biological Resource Centre}, \orgaddress{\state{Paris}, \country{France}}}
\address[6]{\orgdiv{Department of Microbiology, Immunology and Transplantation}, \orgname{Rega Institute for Medical Research, KU Leuven}, \orgaddress{\state{Leuven}, \country{Belgium}}}
\address[7]{\orgdiv{Department of Environmental Science and Technology}, \orgname{University of Maryland}, \orgaddress{\state{Baltimore}, \country{USA}}}

\corresp[$\ast$]{Corresponding author. \href{email:pafilis@hcmr.gr}{pafilis@hcmr.gr}}

\received{Date}{0}{Year}
\revised{Date}{0}{Year}
\accepted{Date}{0}{Year}

%\editor{Associate Editor: Name}

%\abstract{
%\textbf{Motivation:} .\\
%\textbf{Results:} .\\
%\textbf{Availability:} .\\
%\textbf{Contact:} \href{name@email.com}{name@email.com}\\
%\textbf{Supplementary information:} Supplementary data are available at \textit{Journal Name}
%online.}

\abstract{Ecosystem functioning is an integral part of the sciences of climate
    change and conservation. Microbes are known for their versatility, abundance
    and influence on ecosystem functions, but a more thorough analysis of
    ecosystems should include a more complex assembly of organisms.
    For example, plant-arthropod-soil microbiome interactions are important
    associations that have historically been ignored. A synthesized knowledge
    base of biodiversity, in terms of ecological and remote-sensing data remains a
    major challenge. Many worldwide studies have been published regarding soil
    microbiome ecosystems, though there are many gaps to cover the complexity
    of functioning and biodiversity in the local scale. Islands can be
    important case studies for this integration, from micro to macro, with Crete being a great example. 
    Here, we utilize the Island Sampling Day Crete 2016 microbial 16S rRNA gene
    amplicon data and metadata, integrated with soil, faunistic and remote
    sensing data, to decipher the drivers of ecosystem function of the island.
    Cretan macroecology has been studied for centuries for its diverse and
    unique geology, fauna and flora. In addition, Crete as a continental island,
    presents a distinct natural and evolutionary history with high contrasts in
    vegetation cover, climatic conditions and geology. The Island Sampling Day
    Crete 2016 project (co-hosted by the Genomic Standards Consortium (GSC) and
    the Institute of Marine Biology, Biotechnology and Aquaculture (IMBBC-HCMR)), has
    collected microbial 16s amplicon data from 72 distinct, with ecosystem
    diversity topsoil sites from all around Crete, accompanied by FAIR
    (Findable, Accessible, Interoperable and Reproducible) data by design. 
    With this island wide study the GSC put the genomic standards in action.
The preliminary results indicate a notable influence of the pH and the elevation
over the island's microbiota. In particular, sites in higher altitudes found to
be inhabited by a more diverse number of microorganisms, a pattern commonly
seen in several faunistic groups, such as arthropods. These data along with the
open access data regarding arthropod fauna, flora and the vast data of remote
sensing and biodiversity, provide the basis to identify major drivers of
biodiversity, to evaluate hotspots and contribute to foreknowledge of threatened ecosystems.}

\keywords{soil microbiome, island biogeography, Crete, data integration, FAIR data}

% \boxedtext{
% \begin{itemize}
% \item Key boxed text here.
% \item Key boxed text here.
% \item Key boxed text here.
% \end{itemize}}

\maketitle


\section{Introduction}\label{intro}

Soil ecosystems are the cornstone of terrestrial habitats, biodiversity and henceforth human activities.
Complex interactions of multiple systems, nutrient cycling, atmosphere, climate, biogeochemical cycles and biodiversity 
across all forms of life, fauna, flora, bacteria, archaea, fungi, viruses. 
Bacteria and archaea are considered major drivers for the functionality of soil.
They infuence and are infuenced by their environments and their community structure 
is defies their macroscopic functionality.
Global soil microbiome studies \citep{Delgado-Baquerizo2018} and \citep{Bahram2018} and the biogeography \citep{Martiny2006} \citep{Guerra2020}, \citep{Labouyrie2023} yet there are significant blind spots \citep{Guerra2020}. The most resolute study is \citep{Karimi2020} which exemplified the 
remarkable complexity of soil bacterial communuties and the requirement of  dense samplings and isolated systems. 

Island Sampling Day project and the 
Genome Standards Consortium \citep{Field2011} "put standards into action" and the 
18th workshop. 
The choice of an island
Islands are nature's labs \citep{Whittaker2017} and in soil microbes \citep{Li2020} and fungi \citep{Delavaux2021}
Ultimately islands can set the ground to represent ecosystems with data and complex interactions \citep{Davies2016}.
A large scale study in the confined space of the island of Crete, a miniature continent.


The island of Crete, Greece has been studied since the classical times for its' fauna \citep{Sidiropoulos_Polymeni_Legakis_2017,Anastasiou2018Tenebrionid}, flora \citep{Krimbas_2005} and ecosystems \citep{Grove1993}.
Crete is a continental island, fifth largest island of the mediterranean, a mediterranean biodiversity hotspot, model for desertification, climate change \citep{Kougioumoutzis2020} and habitat loss \citep{ISPIKOUDIS1993259},
elevation gradient \citep{trigas2013, FAZAN2017}, grasing impacts on ecosystems \citep{JouffroyBapicot2016} and evolutionary model \citep{POULAKAKIS2002}.
Yet the terrestrial microbial diversity has been unexplored.

Has been sampled from major worldwide projects of microbiome studies \citep{Vasar2022, Labouyrie2023, Bahram2018, Orgiazzi2018}. Soil fungi \citep{Mikryukov2023, Davison2021, Tedersoo2021} and other soil eukaryotes \citep{Aslani2022}


\citep{Vogiatzakis2008_crete}
\citep{Vogiatzakis2016}

\citep{Sfenthourakis2017}
\citep{Vogiatzakis2008_med}

Integration of prior knowledge and 
Genome Standards Consortium \citep{Field2011}, FAIR data \citep{Wilkinson2016}, unrestricted \citep{Amann2019},
rich metadata and platforms of hosting open data like European Soil Data Centre \citep{Panagos2022}

Interactomes as a driver of community composition. Microbial interactions challenges \citep{Faust2021} cohesion? 

In this study, we investigate

%This is an example of a new parapgraph with a numbered footnote\footnote{\url{https://data.gov.uk/}} and a second footnote marker.\footnote{Example of footnote text.}

\section{Materials and Methods}\label{methods}


\subsection{Island Sampling Day: Crete}\label{isd_data}

the data for this study have been deposited in the European Nucleotide Archive (ENA) at EMBL-EBI under accession number PRJEB21776
ENA API \citep{Yuan2023}

\subsection{Data integration}\label{data}

Copernicus CORINE Land Cover \citep{CLC2023}
WorldClim 2.0 \citep{Fick2017}
World Database on Protected Areas (WDPA) \citep{Hanson2022}
Function annotation - FAPROTAX \citep{Louca2016}
Crete geology from IGME
Aridity
%NDVI
Taxonomy Silva 138 \citep{Quast2012}

\subsection{Tools}\label{tools}
dada2 for ASV inferrence \citep{Callahan2016}
PEMA for OTU inferrence \citep{Zafeiropoulos2020}
ape R packege \citep{Paradis2004}
SRS R package \citep{Beule2020}
UMAP ordination \citep{mcinnes2018umap-software}
Numerical ecology analyses with vegan R package \citep{vegan}

Python 3.11.4
Julia Version 1.9.3 \citep{Julia-2017}
R version 4.3.2 \citep{rcoreteam}
Network inferrence using FlashWeave 0.19.2 \citep{Tackmann2019}
Network analysis in igraph R package \citep{Csardi2006}
Differential abundance analysis with ANCOM-BC R package \citep{Lin2020}
Spatial analysis with the sf and stars R packages \citep{Pebesma2023}
U-CIE R package for coloring 3 dimentional data \citep{Koutrouli2022}
ggplot2 \citep{ggplot22016}
pheatmap \citep{Kolde2019}

Computations were performed on HPC infrastrure of HCMR \citep{Zafeiropoulos2021}.

\subsubsection{This is an example for third level head - subsubsection head}\label{subsubsec1}

Lorem ipsum dolor sit amet, consectetur adipiscing elit, sed do
eiusmod tempor incididunt ut labore et dolore magna aliqua. Ut enim ad minim veniam, quis nostrud exercitation ullamco laboris nisi ut aliquip ex ea commodo consequat. %Duis aute irure dolor in reprehenderit in voluptate velit esse cillum dolore eu fugiat nulla pariatur. Excepteur sint occaecat cupidatat non proident, sunt in culpa qui officia deserunt mollit anim id est laborum.

\paragraph{This is an example for fourth level head - paragraph head}

Lorem ipsum dolor sit amet, consectetur adipiscing elit, sed do eiusmod tempor incididunt ut labore et dolore magna aliqua. Ut enim ad minim veniam, quis nostrud exercitation ullamco laboris nisi ut aliquip ex ea commodo consequat. Duis aute irure dolor in reprehenderit in voluptate velit esse cillum dolore eu fugiat nulla pariatur. Excepteur sint occaecat cupidatat non proident, sunt in culpa qui officia deserunt mollit anim id est laborum.

\section{Results}\label{results}

\subsection{This is an example for second level head - subsection head}\label{subsec2}

\subsubsection{This is an example for third level head - subsubsection head}\label{subsubsec2}

Lorem ipsum dolor sit amet, consectetur adipiscing elit, sed do eiusmod tempor incididunt ut labore et dolore magna aliqua. Ut enim ad minim veniam, quis nostrud exercitation ullamco laboris nisi ut aliquip ex ea commodo consequat. Duis aute irure dolor in reprehenderit in voluptate velit esse cillum dolore eu fugiat nulla pariatur. Excepteur sint occaecat cupidatat non proident, sunt in culpa qui officia deserunt mollit anim id est laborum.

\paragraph{This is an example for fourth level head - paragraph head}

Lorem ipsum dolor sit amet, consectetur adipiscing elit, sed do eiusmod tempor incididunt ut labore et dolore magna aliqua. Ut enim ad minim veniam, quis nostrud exercitation ullamco laboris nisi ut aliquip ex ea commodo consequat. Duis aute irure dolor in reprehenderit in voluptate velit esse cillum dolore eu fugiat nulla pariatur. Excepteur sint occaecat cupidatat non proident, sunt in culpa qui officia deserunt mollit anim id est laborum.



\section{Tables}\label{sec5}

Tables can be inserted via the normal table and tabular environment. To put
footnotes inside tables one has to Lorem ipsum dolor sit amet, consectetur adipiscing elit, sed do eiusmod tempor incididunt ut labore et dolore magna aliqua. Ut enim ad minim veniam, quis nostrud exercitation ullamco laboris nisi ut aliquip ex ea commodo consequat. Duis aute irure dolor in reprehenderit in voluptate velit esse cillum dolore eu fugiat nulla pariatur. Excepteur sint occaecat cupidatat non proident, sunt in culpa qui officia deserunt mollit anim id est laborum. use the additional ``tablenotes" environment
enclosing the tabular environment. The footnote appears just below the table
itself (refer Tables~\ref{tab1} and \ref{tab2}).

Lengthy tables which do not fit within textwidth should be set as rotated tables. For this, we need to use \verb+\begin{sidewaystable}...+ \verb+\end{sidewaystable}+ instead of\break \verb+\begin{table}...+ \verb+\end{table}+ environment.


\begin{table}[!t]
\caption{Caption text\label{tab1}}%
\begin{tabular*}{\columnwidth}{@{\extracolsep\fill}llll@{\extracolsep\fill}}
\toprule
column 1 & column 2  & column 3 & column 4\\
\midrule
row 1    & data 1   & data 2  & data 3  \\
row 2    & data 4   & data 5$^{1}$  & data 6  \\
row 3    & data 7   & data 8  & data 9$^{2}$  \\
\botrule
\end{tabular*}
\begin{tablenotes}%
\item Source: This is an example of table footnote this is an example of table footnote this is an example of table footnote this is an example of~table footnote this is an example of table footnote
\item[$^{1}$] Example for a first table footnote.
\item[$^{2}$] Example for a second table footnote.
\end{tablenotes}
\end{table}

\begin{table*}[t]
\caption{Example of a lengthy table which is set to full textwidth.\label{tab2}}
\tabcolsep=0pt%%
\begin{tabular*}{\textwidth}{@{\extracolsep{\fill}}lcccccc@{\extracolsep{\fill}}}
\toprule%
& \multicolumn{3}{@{}c@{}}{Element 1$^{1}$} & \multicolumn{3}{@{}c@{}}{Element 2$^{2}$} \\
\cline{2-4}\cline{5-7}%
Project & Energy & $\sigma_{calc}$ & $\sigma_{expt}$ & Energy & $\sigma_{calc}$ & $\sigma_{expt}$ \\
\midrule
Element 3  & 990 A & 1168 & $1547\pm12$ & 780 A & 1166 & $1239\pm100$\\
Element 4  & 500 A & 961  & $922\pm10$  & 900 A & 1268 & $1092\pm40$\\
\botrule
\end{tabular*}
\begin{tablenotes}%
\item Note: This is an example of table footnote this is an example of table footnote this is an example of table footnote this is an example of~table footnote this is an example of table footnote
\item[$^{1}$] Example for a first table footnote.
\item[$^{2}$] Example for a second table footnote.\vspace*{6pt}
\end{tablenotes}
\end{table*}

\section{Figures}\label{sec6}

As per display \LaTeX\ standards one has to use eps images for \verb+latex+ compilation and \verb+pdf/jpg/png+ images for
\verb+pdflatex+ compilation. This is one of the major differences between \verb+latex+
and \verb+pdflatex+. The images should be single-page documents. The command for inserting images
for \verb+latex+ and \verb+pdflatex+ can be generalized. The package used to insert images in \verb+latex/pdflatex+ is the
graphicx package. Figures can be inserted via the normal figure environment as shown in the below example:


\begin{figure}[!t]%
\centering
{\color{black!20}\rule{213pt}{37pt}}
\caption{This is a widefig. This is an example of a long caption this is an example of a long caption  this is an example of a long caption this is an example of a long caption}\label{fig1}
\end{figure}

\begin{figure*}[!t]%
\centering
{\color{black!20}\rule{438pt}{74pt}}
\caption{This is a widefig. This is an example of a long caption this is an example of a long caption  this is an example of a long caption this is an example of a long caption}\label{fig2}
\end{figure*}

For sample purposes, we have included the width of images in the
optional argument of \verb+\includegraphics+ tag. Please ignore this.
Lengthy figures which do not fit within textwidth should be set in rotated mode. For rotated figures, we need to use \verb+\begin{sidewaysfigure}+ \verb+...+ \verb+\end{sidewaysfigure}+ instead of the \verb+\begin{figure}+ \verb+...+ \verb+\end{figure}+ environment.



With standard numerical .bst files, only numerical citations are possible.
With an author-year .bst file, both numerical and author-year citations are possible.



\noindent
Multiple citations as normal:


\section{Discussion}\label{discussion}

Networks specialists generalists \citep{Barberan2012}

Elevation gradients of biodiversity are known since Humboldt's work \citep{Rahbek2019} 
yet these patterns remain elusive regarding the soil microbiome \citep{Looby2020, Siles2023}.
Mostly because it's difficult to isolate other cofounding effects \citep{Nottingham2018}. 

Functions of Richtis gorge are a signal in need to be further investigated.

Enumeratic all biodivarsity \citep{Anthony2023} and decipher the biogeochemical 
processes and interactions of fauna, flora and microbes \citep{Fry2019, Crowther2019,GRANDY201640,Delgado-Baquerizo2020}

Global hotspots \citep{Guerra2022}
and soil conservation instead of specific species \citep{Guerra2021}
and policy \citep{KONINGER2022} across countries \citep{Putten2023}


\section{Conclusion}

Mountain peaks soil microbes \citep{Adamczyk2019}
These studies can act as early warning signals towards public health conserns \citep{Banerjee2023}.
Some Conclusions here.

%%%%%%%%%%%%%%

%\begin{appendices}
%
%\section{Appendix 1}\label{appendix1}
%
%Nam dui ligula, fringilla a, euismod sodales, sollicitudin vel, wisi. Morbi auctor lorem non justo. Nam lacus libero,
%pretium at, lobortis vitae, ultricies et, tellus. Donec aliquet, tortor sed accumsan bibendum, erat ligula aliquet magna,
%vitae ornare odio metus a mi. Morbi ac orci et nisl hendrerit mollis. Suspendisse ut massa. Cras nec ante. Pellentesque
%a nulla. Cum sociis natoque penatibus et magnis dis parturient montes, nascetur ridiculus mus. Aliquam tincidunt
%urna. Nulla ullamcorper vestibulum turpis. Pellentesque cursus luctus mauris.
%
%
%\section{Appendix 2}\label{appendix2}
%\subsubsection{Subsubsection title of first appendix}\label{subsubsec3}
%
%Fusce mauris. Vestibulum luctus nibh at lectus. Sed bibendum, nulla a faucibus semper, leo velit ultricies tellus, ac
%venenatis arcu wisi vel nisl. Vestibulum diam. Aliquam pellentesque, augue quis sagittis posuere, turpis lacus congue
%quam, in hendrerit risus eros eget felis.
%
%\end{appendices}

\section{Competing interests}
No competing interest is declared.

\section{Author contributions statement}

Must include all authors, identified by initials, for example:
S.R. and D.A. conceived the experiment(s),  S.R. conducted the experiment(s), S.R. and D.A. analysed the results.  S.R. and D.A. wrote and reviewed the manuscript.

\section{Acknowledgments}
The authors thank the anonymous reviewers for their valuable suggestions. This work is supported in part by funds from the National Science Foundation (NSF: \# 1636933 and \# 1920920).


\bibliographystyle{unsrt}
\bibliography{reference}


%USE THE BELOW OPTIONS IN CASE YOU NEED AUTHOR YEAR FORMAT.
%\bibliographystyle{abbrvnat}
%\bibliography{reference}


\end{document}
