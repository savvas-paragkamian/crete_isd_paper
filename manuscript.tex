%%
%% Copyright 2022 OXFORD UNIVERSITY PRESS
%%
%% This file is part of the 'oup-authoring-template Bundle'.
%% ---------------------------------------------
%%
%% It may be distributed under the conditions of the LaTeX Project Public
%% License, either version 1.2 of this license or (at your option) any
%% later version.  The latest version of this license is in
%%    http://www.latex-project.org/lppl.txt
%% and version 1.2 or later is part of all distributions of LaTeX
%% version 1999/12/01 or later.
%%
%% The list of all files belonging to the 'oup-authoring-template Bundle' is
%% given in the file `manifest.txt'.
%%
%% Template article for OXFORD UNIVERSITY PRESS's document class `oup-authoring-template'
%% with bibliographic references
%%

%%%CONTEMPORARY%%%
\documentclass[unnumsec,webpdf,contemporary,large]{oup-authoring-template}%
%\documentclass[unnumsec,webpdf,contemporary,large,namedate]{oup-authoring-template}% uncomment this line for author year citations and comment the above
%\documentclass[unnumsec,webpdf,contemporary,medium]{oup-authoring-template}
%\documentclass[unnumsec,webpdf,contemporary,small]{oup-authoring-template}

%%%MODERN%%%
%\documentclass[unnumsec,webpdf,modern,large]{oup-authoring-template}
%\documentclass[unnumsec,webpdf,modern,large,namedate]{oup-authoring-template}% uncomment this line for author year citations and comment the above
%\documentclass[unnumsec,webpdf,modern,medium]{oup-authoring-template}
%\documentclass[unnumsec,webpdf,modern,small]{oup-authoring-template}

%%%TRADITIONAL%%%
%\documentclass[unnumsec,webpdf,traditional,large]{oup-authoring-template}
%\documentclass[unnumsec,webpdf,traditional,large,namedate]{oup-authoring-template}% uncomment this line for author year citations and comment the above
%\documentclass[unnumsec,namedate,webpdf,traditional,medium]{oup-authoring-template}
%\documentclass[namedate,webpdf,traditional,small]{oup-authoring-template}

%\onecolumn % for one column layouts

%\usepackage{showframe}
\usepackage[T1]{fontenc}
\graphicspath{{Fig/}}

% line numbers
%\usepackage[mathlines, switch]{lineno}
%\usepackage[right]{lineno}

\theoremstyle{thmstyleone}%
\newtheorem{theorem}{Theorem}%  meant for continuous numbers
%%\newtheorem{theorem}{Theorem}[section]% meant for sectionwise numbers
%% optional argument [theorem] produces theorem numbering sequence instead of independent numbers for Proposition
\newtheorem{proposition}[theorem]{Proposition}%
%%\newtheorem{proposition}{Proposition}% to get separate numbers for theorem and proposition etc.
\theoremstyle{thmstyletwo}%
\newtheorem{example}{Example}%
\newtheorem{remark}{Remark}%
\theoremstyle{thmstylethree}%
\newtheorem{definition}{Definition}

\begin{document}

\journaltitle{The ISME Journal}
\DOI{DOI HERE}
\copyrightyear{2024}
\pubyear{2024}
\access{Advance Access Publication Date: Day Month Year}
\appnotes{Paper}

\firstpage{1}

%\subtitle{Subject Section}

\title[Crete's soil microbiome]{Exploring the Soil Microbiome coupling with the Crete island Data Cube}

\author[1,2]{Savvas Paragkamian\ORCID{0000-0002-8508-2521}}
\author[3,4]{Johanna B. Holm\ORCID{0000-0002-7646-4085}}
\author[5]{Christina Pavloudi\ORCID{0000-0001-5106-6067}}
\author[6]{Haris Zafeiropoulos\ORCID{0000-0002-4405-6802}}
\author[1]{Christos A. Christakis\ORCID{0000-0002-7075-0996}}
\author[1]{Melanthia Stavroulaki\ORCID{0000-0003-4392-7159}}
\author[1]{Dimitris Tsaparis\ORCID{0000-0001-9626-5553}}
%\author[1]{Iraklis Vretzakis\ORCID{0000-0001-9626-5553}}
\author[7]{Stephanie A. Yarwood\ORCID{0000-0003-4593-7908}}
\author[2]{Antonios Magoulas}
\author[1]{Panagiotis F. Sarris\ORCID{0000-0001-7000-8997}}
\author[2]{Giorgos Kotoulas\ORCID{0000-0002-4667-8533}}
\author[3]{Lynn Schriml\ORCID{0000-0001-8910-9851}}
\author[2,$\ast$]{Evangelos Pafilis\ORCID{0000-0001-5079-0125}}

\authormark{Paragkamian et al.}

\address[1]{\orgdiv{Department of Biology}, \orgname{University of Crete}, \orgaddress{\street{Voutes}, \postcode{70013}, \state{Crete}, \country{Greece}}}
\address[2]{\orgdiv{Institute of Marine Biology, Biotechnology and Aquaculture (IMBBC)}, \orgname{Hellenic Centre for Marine Research (HCMR)}, \orgaddress{\street{Gournes}, \postcode{71500}, \state{Crete}, \country{Greece}}}
\address[3]{\orgdiv{Institute for Genome Sciences}, \orgname{University of Maryland School of Medicine}, \orgaddress{\state{Baltimore}, \country{USA}}}
\address[4]{\orgdiv{Department of Microbiology and Immunology}, \orgname{University of Maryland School of Medicine}, \orgaddress{\state{Baltimore}, \country{USA}}}
\address[5]{\orgname{European Marine Biological Resource Centre}, \orgaddress{\state{Paris}, \country{France}}}
\address[6]{\orgdiv{Department of Microbiology, Immunology and Transplantation}, \orgname{Rega Institute for Medical Research, KU Leuven}, \orgaddress{\state{Leuven}, \country{Belgium}}}
\address[7]{\orgdiv{Department of Environmental Science and Technology}, \orgname{University of Maryland}, \orgaddress{\state{Baltimore}, \country{USA}}}

\corresp[$\ast$]{Corresponding author. \href{email:pafilis@hcmr.gr}{pafilis@hcmr.gr}}

\received{Date}{0}{Year}
\revised{Date}{0}{Year}
\accepted{Date}{0}{Year}

%\editor{Associate Editor: Name}

%\abstract{
%\textbf{Motivation:} .\\
%\textbf{Results:} .\\
%\textbf{Availability:} .\\
%\textbf{Contact:} \href{name@email.com}{name@email.com}\\
%\textbf{Supplementary information:} Supplementary data are available at \textit{Journal Name}
%online.}

\abstract{Ecosystem functioning is an integral part of the sciences of climate
    change and conservation. Microbes are known for their versatility, abundance
    and influence on ecosystem functions, but a more thorough analysis of
    ecosystems should include a more complex assembly of organisms.
    For example, plant-arthropod-soil microbiome interactions are important
    associations that have historically been ignored. A synthesized knowledge
    base of biodiversity, in terms of ecological and remote-sensing data remains a
    major challenge. Many worldwide studies have been published regarding soil
    microbiome ecosystems, though there are many gaps to cover the complexity
    of functioning and biodiversity in the local scale. Islands can be
    important case studies for this integration, from micro to macro, with Crete being a great example. 
    Here, we utilize the Island Sampling Day Crete 2016 microbial 16S rRNA gene
    amplicon data and metadata, integrated with soil, faunistic and remote
    sensing data, to decipher the drivers of ecosystem function of the island.
    Cretan macroecology has been studied for centuries for its diverse and
    unique geology, fauna and flora. In addition, Crete as a continental island,
    presents a distinct natural and evolutionary history with high contrasts in
    vegetation cover, climatic conditions and geology. The Island Sampling Day
    Crete 2016 project (co-hosted by the Genomic Standards Consortium (GSC) and
    the Institute of Marine Biology, Biotechnology and Aquaculture (IMBBC-HCMR)), has
    collected microbial 16s amplicon data from 72 distinct, with ecosystem
    diversity topsoil sites from all around Crete, accompanied by FAIR
    (Findable, Accessible, Interoperable and Reproducible) data by design. 
    With this island wide study the GSC put the genomic standards in action.
The preliminary results indicate a notable influence of the pH and the elevation
over the island's microbiota. In particular, sites in higher altitudes found to
be inhabited by a more diverse number of microorganisms, a pattern commonly
seen in several faunistic groups, such as arthropods. These data along with the
open access data regarding arthropod fauna, flora and the vast data of remote
sensing and biodiversity, provide the basis to identify major drivers of
biodiversity, to evaluate hotspots and contribute to foreknowledge of threatened ecosystems.}

\keywords{soil microbiome, networks, island biogeography, Crete, data integration, FAIR data}

% \boxedtext{
% \begin{itemize}
% \item Key boxed text here.
% \item Key boxed text here.
% \item Key boxed text here.
% \end{itemize}}

\maketitle

\section{Introduction}\label{intro}

Soil ecosystems are the cornerstone of terrestrial habitats, biodiversity and henceforth human activities.
Soils are characterised by multiple properties; chemical, physical and biological that 
form complex interdependant interactions. Biodiversity of soils covers
all forms of life, fauna, flora, bacteria, archaea, fungi, viruses. 
Bacteria and archaea are considered major drivers for the functionality of soil.
They infuence and are infuenced by their environment and their community structure 
defies their macroscopic functionality.
Global soil microbiome studies have been employed to decipher soil microbiome
compositions \cite{thompson2017a-communal, Delgado-Baquerizo2018, Labouyrie2023},
functions \cite{Bahram2018} and biogeography \cite{Martiny2006, Guerra2020}.
These results showed the remarkable diversity in soils yet there are blind spots \cite{Guerra2020}
and these sampling are sparce when considering samples per area density. One of most resolute
study is by \cite{Karimi2020} which exemplified the 
vast complexity of soil bacterial communuties and the requirement of dense samplings and isolated systems.

Islands are nature's labs \cite{Whittaker2017} because of their isolation and smaller scale.
Borrowing this paradigm, soil microbes \cite{Li2020} and mycorrhizal fungi \cite{Delavaux2021} studies
in islands show the benefits of using islands as models. Ultimately islands canr
set the ground to represent ecosystems with data and complex interactions \cite{Davies2016}.
This was also the case for the Island Sampling Day (ISD) project \cite{holm2024}
of the Genome Standards Consortium \cite{Field2011}
during the 18th workshop in June 2026 in Crete island, Greece. The goal was to "put standards into action"
in a soil microbiome survey with a dense sampling, 0.017 samples per km\textsuperscript{2}.
Hence, ISD is a large scale study in the confined space of the island of Crete which 
is considered a miniature continent \cite{Vogiatzakis2008_crete}.

Crete is a continental forarc island \cite{ali2016}, fifth largest island of the mediterranean (~8300 km\textsuperscript{2}),
and a mediterranean biodiversity hotspot \cite{myers2000biodiversity}.
The island of Crete has been studied since the classical times for its'
fauna \cite{Sidiropoulos_Polymeni_Legakis_2017,Anastasiou2018Tenebrionid}, flora \cite{Krimbas_2005} and ecosystems \cite{Grove1993}.
Crete is home to the only endemic mammal of Greece, the Cretan shrew (\textit{Crocidura zimmermanni}),
more than 350 endemic arthropods \cite{bolanakis2024} and 183 endemic plants \cite{Kougioumoutzis2020}
among them a tree \textit{Zelkova abelicea}. Multifacet factors have shaped the
biodiversity of the island, for example the sharp elevation gradient \cite{trigas2013, FAZAN2017},
the complex evolutionary history \cite{POULAKAKIS2002} and the human - nature
interactions over thousands of years \cite{Vogiatzakis2008_med, Sfenthourakis2017}.
The major threats of human activities are becoming apperend in the island's ecosystems,
like desertification \cite{KARAMESOUTI2018266}, intesive grasing \cite{JouffroyBapicot2016},
climate change \cite{Kougioumoutzis2020,Vogiatzakis2016} and habitat loss \cite{ISPIKOUDIS1993259}.
Yet the topsoil microbial diversity of Crete has been unexplored.

Worldwide projects of microbiome studies have collected one or two topsoil
samples from Crete \cite{Vasar2022, Labouyrie2023, Bahram2018, Orgiazzi2018}.
Some have focused on soil fungi \cite{Mikryukov2023, Davison2021, Tedersoo2021}
and other soil eukaryotes \cite{Aslani2022}.
The only thorough microbiome study of a soil ecosystem in Crete, to our knowledge,
is in the north west part of the island, the Koiliaris Critical Zone Observatory \cite{tsiknia2014}.
Apart from the soil microbiome, soil physical and chemical properties has been
investigaded by globan and european projects like the Forum of European Geological Surveys
(FOREGS) \cite{nerc19017}, the Geochemical Mapping of Agricultural and Grazing Land
Soil in Europe (GEMAS) \cite{REIMANN2018302} and the Soil Profile Analytical
Database for Europe (SPADE) \cite{Hiederer2006}.

Data and metadata of these samplings are stored in different databases, yet 
great effort from these distinct communities have led to establishing standards
to enable FAIR data \cite{Wilkinson2016}. For amplicon sequences the Genome Standards
Consortium \cite{Field2011} has established the MIMARKS \cite{yilmaz2011minimum}
standards and has been a advocate for open and unrestricted data \cite{Amann2019}.
For soil data, rich metadata and platforms of hosting open data exist, e.g the 
European Soil Data Centre \cite{Panagos2022} and the World Soil Information
Service (WoSIS) \cite{Batjes2024}. These platforms also provide maps.

Spatial and remote sensing data

Soil structure and chemistry, holistic

Integration of prior knowledge and 


Interactomes have been characterised as a driver of community composition.
Microbial interactions challenges \cite{Faust2021} cohesion? 

In this study, we integrade multiple types of data and methods to decipher hidden 
signals of the topsoil bacterial communities of Crete. 


\section{Materials and Methods}\label{methods}


\subsection{Island Sampling Day: Crete}\label{isd_data}

The data for this study have been deposited in the European Nucleotide Archive (ENA) at EMBL-EBI under accession number PRJEB21776
ENA API \cite{Yuan2023}

\subsection{Data integration}\label{data}

Copernicus CORINE Land Cover \cite{CLC2023}
WorldClim 2.0 \cite{Fick2017}
World Database on Protected Areas (WDPA) \cite{Hanson2022}
Function annotation - FAPROTAX \cite{Louca2016}
Crete geology from IGME
Aridity
%NDVI
Taxonomy Silva 138 \cite{Quast2012}

\subsection{Tools}\label{tools}
dada2 for ASV inferrence \cite{Callahan2016}
PEMA for OTU inferrence \cite{Zafeiropoulos2020}
ape R packege \cite{Paradis2004}
SRS R package \cite{Beule2020}
UMAP ordination \cite{mcinnes2018umap-software}
Numerical ecology analyses with vegan R package \cite{vegan}

Python 3.11.4
Julia Version 1.9.3 \cite{Julia-2017}
R version 4.3.2 \cite{rcoreteam}
Network inferrence using FlashWeave 0.19.2 \cite{Tackmann2019}
Network analysis in igraph R package \cite{Csardi2006}
Differential abundance analysis with ANCOM-BC R package \cite{Lin2020,Lin2023}
Spatial analysis with the sf and stars R packages \cite{Pebesma2023}
U-CIE R package for coloring 3 dimentional data \cite{Koutrouli2022}
ggplot2 \cite{ggplot22016}
pheatmap \cite{Kolde2019}

Computations were performed on HPC infrastrure of HCMR \cite{Zafeiropoulos2021}.

\section{Results}\label{results}

\subsection{Soil microbiome}\label{soil_microbiome}

Summary of the pipeline results

Generalists and specialists


\subsection{Crete Data Cube}\label{data_cube}

\subsection{Community drivers}\label{Drivers}

\subsection{Interactome}\label{interactome}

\subsection{Human interventions}\label{Human interventions}

\subsubsection{Functions}\label{functions}

\subsubsection{Significant taxa}\label{sig_taxa}

Differential abundance

\section{Discussion}\label{discussion}

Networks specialists generalists \cite{Barberan2012}

Elevation gradients of biodiversity are known since Humboldt's work \cite{Rahbek2019} 
yet these patterns remain elusive regarding the soil microbiome \cite{Looby2020, Siles2023}.
Mostly because it's difficult to isolate other cofounding effects \cite{Nottingham2018}. 

Functions of Richtis gorge are a signal in need to be further investigated.

Enumeratic all biodivarsity \cite{Anthony2023} and decipher the biogeochemical 
processes and interactions of fauna, flora and microbes \cite{Fry2019, Crowther2019,GRANDY201640,Delgado-Baquerizo2020}

Global hotspots \cite{Guerra2022}
and soil conservation instead of specific species \cite{Guerra2021}
and policy \cite{KONINGER2022} across countries \cite{Putten2023}

"A holistic perspective on soil architecture is needed as a key to soil functions"
\section{Conclusion}

Mountain peaks soil microbes \cite{Adamczyk2019}
Even though amplicon studies in soil should be interpeted with caution \cite{alteio2021} they 
can act as early warning signals towards public health conserns \cite{Banerjee2023}.

Future works, shotgun metagenomics, long reads, need for even higher resolution samplings. 



%%%%%%%%%%%%%%

%\begin{appendices}
%
%\section{Appendix 1}\label{appendix1}
%
%Nam dui ligula, fringilla a, euismod sodales, sollicitudin vel, wisi. Morbi auctor lorem non justo. Nam lacus libero,
%pretium at, lobortis vitae, ultricies et, tellus. Donec aliquet, tortor sed accumsan bibendum, erat ligula aliquet magna,
%vitae ornare odio metus a mi. Morbi ac orci et nisl hendrerit mollis. Suspendisse ut massa. Cras nec ante. Pellentesque
%a nulla. Cum sociis natoque penatibus et magnis dis parturient montes, nascetur ridiculus mus. Aliquam tincidunt
%urna. Nulla ullamcorper vestibulum turpis. Pellentesque cursus luctus mauris.
%
%
%\section{Appendix 2}\label{appendix2}
%\subsubsection{Subsubsection title of first appendix}\label{subsubsec3}
%
%Fusce mauris. Vestibulum luctus nibh at lectus. Sed bibendum, nulla a faucibus semper, leo velit ultricies tellus, ac
%venenatis arcu wisi vel nisl. Vestibulum diam. Aliquam pellentesque, augue quis sagittis posuere, turpis lacus congue
%quam, in hendrerit risus eros eget felis.
%
%\end{appendices}

\section{Competing interests}
No competing interest is declared.

\section{Author contributions statement}

Must include all authors, identified by initials, for example:
S.R. and D.A. conceived the experiment(s),  S.R. conducted the experiment(s), S.R. and D.A. analysed the results.  S.R. and D.A. wrote and reviewed the manuscript.

\section{Acknowledgments}
The authors thank the anonymous reviewers for their valuable suggestions. This work is supported in part by funds from the National Science Foundation (NSF: \# 1636933 and \# 1920920).


\bibliographystyle{unsrt}
\bibliography{reference}


%USE THE BELOW OPTIONS IN CASE YOU NEED AUTHOR YEAR FORMAT.
%\bibliographystyle{abbrvnat}
%\bibliography{reference}


\end{document}
